\section*{Abstract}

Defining optimal forest management strategies in a changing world is a challenge in the field of forest ecology. The complexity of forest ecosystems, coupled with the uncertainty of future climate conditions, makes it difficult to determine the best course of action. The concept of forest diversity is a key consideration in this debate, as it is believed to be a significant factor in the resilience of forest ecosystems. This study aims to explore how diversity can be used as a management tool to maintain the ecosystem in a desirable state. To this end, we will use a theoretical model of a mixed-species, multi-layered forest, and apply control theory and viability analyses to assess the relationship between diversity and management trajectories, considering both species and vertical diversity at the stand level. Our results showed that diversity and extraction constraints were not compatible. However the viability approach allowed to define viable trajectories for each of them. If the method may need further development in order to be applied to real forest management, this study already provides a first step, and gives insight on the way forward.

\noindent \textbf{Keywords:} forest management, diversity, control theory, viability theory