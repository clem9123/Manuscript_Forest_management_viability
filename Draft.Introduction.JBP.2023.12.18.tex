\documentclass{article}

%----------------------------------------------------------------------------------------
%	PACKAGES AND OTHER DOCUMENT CONFIGURATIONS
%----------------------------------------------------------------------------------------

\usepackage[T1]{fontenc}
\usepackage[utf8]{inputenc}
\usepackage{lmodern}
\usepackage[english]{babel}
\usepackage[autostyle]{csquotes}
\usepackage{graphicx} % Required for inserting images
\usepackage[hyphens]{url}
\usepackage{hyperref} % Required for hyperlinks
\usepackage{amsmath}      % for additional mathematical features
\usepackage{amsfonts}     % for mathbb command
\usepackage{float} 	  % for improved float features
\usepackage{setspace}    % for \doublespacing command
\doublespacing
\usepackage{rotating}   % for sidewaysfigure

\usepackage[backend=biber,style=authoryear, uniquename=false]{biblatex}
\renewcommand*{\bibfont}{\normalfont\small}
\addbibresource{Bibliography.bib}
\usepackage{tcolorbox}
\usepackage{amssymb} 

%----------------------------------------------------------------------------------------
%	DOCUMENT MARGINS
%----------------------------------------------------------------------------------------

\usepackage{geometry} % Required for adjusting page dimensions and margins

\geometry{
	paper=a4paper, % Paper size, change to letterpaper for US letter size
	top=3cm, % Top margin
	bottom=3cm, % Bottom margin
	left=3cm, % Left margin
	right=3cm, % Right margin
	headheight = 10pt, % Header height
	footskip = 1.5cm, % Space from the bottom margin to the baseline of the footer
	headsep = 1.2cm, % Space from the top margin to the baseline of the header
	%showframe, % Uncomment to show how the type block is set on the page
}

%----------------------------------------------------------------------------------------
%	SECTION TITLE APPEARANCE
%----------------------------------------------------------------------------------------


\title{Diversity as a management tool for forest ecosystem services \\ 
\large A theoretical exploration using control theory and viability analyses}
\author{Clementine de Montgolfier}
\date{Dec 2023}

\begin{document}

\maketitle

\section{Summary}

Defining optimal forest management strategies in a changing world is a challenge in the field of forest ecology. The complexity of forest ecosystems, coupled with the uncertainty of future climate conditions, makes it difficult to determine the best course of action. The concept of forest diversity is a key consideration in this debate, as it is believed to be a significant factor in the resilience of forest ecosystems. This study aims to explore whether diversity can be used as a management tool to maintain ecosystem services. To this end, we will use a theoretical model of a mixed-species, multi-layered forest, and apply control theory and viability analyses to assess the relationship between diversity and management trajectories, considering both species and vertical diversity at the stand level.\\

\noindent \textbf{Keywords:} forest management, diversity, control theory, viability theory

\section{Intro}

Forest health metrics show rapid changes in the forest composition, structure, dynamics and the provisioning of ecosystem services to human populations across the world \citep{REFS}. Managers have engaged adaptative policy actions to robustly sustain the state of these forests and their ecosystem services. One class of strategy is now frequently compared empirically and with modesl, involving the adaptation from monospecific forets stand into mixed species stands or uneven forest management by replacing clear-cutting by retention forestry and irregular shelterwood involving a diversity of selectives logging practices or targets \citep{Raymond2009}. Knowledge of these two practices is still limited and the results are not always consensual but they are already implemented and rely on the idea that diversification is a way to increase multiple ecosystem services. But given the many uncertainties in climate and complex response of forest systems to management actions, as well as the extremely large number of possible strategies of diversification involved, principle theories, models and decision science are frequently called upon to assist their design \citep{REFS}. Particularly to understand the link between climate change, biodiversity, levels of a diversity of ecosystem services and and management practices (i.e. disturbance regimes). \\

- Raymond, P., Bédard, S., Roy, V., Larouche, C., and Tremblay, S. (2009). The irregular shelter-
wood system: Review, classification, and potential application to forests affected by partial
disturbances. Journal of Forestry, 107:405–413.

\subsection{Knowledge on the relationship between biodiversity, ecosystem services and management practices in mixed-species stands}


\subsection{link between forest biodiversity and ecosystem multi-functionality}

forest biodiversity is managed to produced simultaneously multiple services/functions. 

- multi-functional index (MFI), $H_{ES}$ diversity of ecosystem services, $H_{species composition, structural, functional traits}$: diversity of species composition, of tree life-history/functional traits, forest structural traits. examples of diversity of services: regulation of flood, carbon storage and also cultural and aesthetic values, etc ....

- hypothesis: $MFI = H_{ES} = f(H_{biodiv})? $

- Garland, G., Banerjee, S., Edlinger, A., Miranda Oliveira, E., Herzog, C., Wittwer, R., Philippot,
L., Maestre, F. T., and van Der Heijden, M. G. (2021). A closer look at the functions behind
ecosystem multifunctionality: A review. Journal of Ecology, 109(2):600–613.
- Meyer, S. T., Ptacnik, R., Hillebrand, H., Bessler, H., Buchmann, N., Ebeling, A., Eisenhauer,
N., Engels, C., Fischer, M., Halle, S., et al. (2018). Biodiversity–multifunctionality relation-
ships depend on identity and number of measured functions. Nature ecology & evolution,
2(1):44–49.
- Schuldt, A., Assmann, T., Brezzi, M., Buscot, F., Eichenberg, D., Gutknecht, J., Härdtle, W.,
He, J.-S., Klein, A.-M., Kühn, P., et al. (2018). Biodiversity across trophic levels drives
multifunctionality in highly diverse forests. Nature communications, 9(1):1–10.
- Pasari, J. R., Levi, T., Zavaleta, E. S., and Tilman, D. (2013). Several scales of biodiversity affect
ecosystem multifunctionality. Proceedings of the National Academy of Sciences, 110(25):10219–
10222.
- Brandt, P., Abson, D. J., DellaSala, D. A., Feller, R., and von Wehrden, H. (2014). Mul-
tifunctionality and biodiversity: Ecosystem services in temperate rainforests of the pacific
northwest, usa. Biological Conservation, 169:362–371.

- Blindly increasing biodiversity can increase forest MFI without optimizing any of them and leading sometimes to trade-offs \citep{vanderplasJackofalltradesEffectsDrive2016}, equivalent to the parabole of the jack-of-all-trades \& master of none. For instance, tade-offs can be observed in the balance between young and old grown forests, if the former are more productive, they store less carbon than the latter ~\autocite{caspersenSuccessionalDiversityForest2001}. 
- => first problem of understanding the relationship between biodiversity and the level of every ecosystem services in different conditions

\subsection{Complex relationship between biodiversity and the level of specific ecosystem services}

(1): relationship between biodiversity, productivity and forest biomass: 

(i) Relationship between biodiversity - ecosystem services (BES: NB: think IPBES) is equivocal:

(ii) positive relationships: in grasslands \citep{tilmanBiodiversityPopulationEcosystem1996} in forests  \citep{liangPositiveBiodiversityproductivityRelationship2016, morinTreeSpeciesRichness2011,paquetteEffectBiodiversityTree2011,jourdanManagingMixedStands2021}.

(iii) negative or no relationships: many refs in grassland \citep{REFs} and especially forests  \citep{forresterReviewProcessesDiversity2016}.

(iv) causes of discrepancy and context dependence. (a) Depends bio-physical factors, such as competitive exclusion, niche complementary, soil and climate conditions for species diversity \citep{juckerClimateModulatesEffects2016, Pichancourt2014, Pichancourt2023}, and disturbance regimes \citep{Pichancourt2014, Pichancourt2023}. (b) Depends also on epistemological factors such as the sampling design \citep{aliBiodiversityEcosystemFunctioning2023}, and the type of diversity indicator that can lead to contradictory results such as when using tree species richness \citep{juckerClimateModulatesEffects2016}, tree functional traits \citep{Pichancourt2014, Pichancourt2023}, diversity of other forest taxa \citep{Runting2019}, indicators of vertical structural diversity \citep{guldinRoleUnevenAgedSilviculture1996, noletComparingEffectsEven2018}. 

Runting, R. K., Ruslandi, Griscom, B., Struebig, M. J., Satar, M., Meijaard, E., Burivalova, Z.,
Cheyne, S., Deere, N. J., Game, E., Putz, F., Wells, J. A., Wilting, A., Ancrenaz, M., Ellis,
P., Khan, F., Leavitt, S. M., Marshall, A., Possingham, H., Watson, J., and Venter, O. (2019).
Larger gains from improved management over sparing–sharing for tropical forests. Nature
Sustainability, 2:53–61.

\subsection{Link between natural and controlled disturbance regimes on biodiversity & ecosystem services}

(1)timber extraction is a special service as it involves management disturbances such as planting and logging, in order to selectively reduce forest biomass on the short term, to then drive the selective increase in the productivity of other species on the long-term. => understand the link between the control of management disturbances, biodiversity and ecosystem services.

(2) Types of management disturbances? planting and logging

(3) 4 hypotheses in the literature: (i) increased intensity of planting/logging, (ii) increased frequency of planting/logging, (iii)increased selective logging on most dominant species and selective planting of less productive species, (iv) increased diversity of selective logging/planting

- Barabás, G., D’Andrea, R., and Stump, S. (2018). Chesson’s coexistence theory. Ecological
Monographs, 88:277–303.
- Chesson, P. (2000). Mechanisms of maintenance of species diversity. Annual review of Ecology
and Systematics, pages 343–366.
- Chesson, P. (2018). Updates on mechanisms of maintenance of species diversity. Journal of
ecology, 106(5):1773–1794.
- Fox, J. (2013). The intermediate disturbance hypothesis should be abandoned. Trends in ecology
& evolution, 28 2:86–92.
- Pichancourt, J.-B., Firn, J., Chades, I., and Martin, T. (2014). Growing biodiverse carbon-rich
forests. Global change biology, 20 2:382–93.
- Pichancourt, J.-B. (2023). Some fundamental elements for studying social-ecological co-existence
in forest common pool resources. PeerJ, 11:e14731.
- Raymond, P., Bédard, S., Roy, V., Larouche, C., and Tremblay, S. (2009). The irregular shelter-
wood system: Review, classification, and potential application to forests affected by partial
disturbances. Journal of Forestry, 107:405–413.

(4) these studies lead to several recommendations to control the level of biological diversity through management disturbances, in order to provide guidelines for shelterwood practices in mixed-species stands: 

(i) maximizing the extraction of the most dominant species at a given time can increase tree functional diversity \citep{Pichancourt2014}.
(ii) intermediate-to-maximal diversity of selective logging targets can increase tree functional diversity  \citep{Pichancourt2023}, providing guidel
(iii) intermediate intensity or frequency on all species has no interest \citep{Chesson2000, Fox2013, Barabas2018}
(iv) but issues when controlling the spill-over effect on ecosystem services \citep{Raymond2009, Pichancourt2014, Pichancourt2023}

\subsection{The interest of control theory for exploring and discovering }

(1) Control theory, blablabla ... \citep{REF}
(2) some methods, such as the viability theory \citep{Aubin1990, Aubin2011}, have been developed to actively find at least one sequence of actions that sustain multiple objectives within the constraints of satisfaction. 
(3) Viability has been applied when objectives are ecosystem services in forest social-ecological systems \citep{mathiasUsingViabilityTheory2015, Houballah2021, Houballah2023}. In these studies, biodiversity was viewed as  used to find a set of viable control sequences to sustain forest biodiversity and some ecosystem services through the control of timber extraction. 
(4) However it has never been used to understand how controlling the diversity species and selective disturbance targets can lead to viable levels MFI index and biological diversity, therefore matching our original question of controlling mixed species stands through complex shelterwood practices, even less when considering the objective of adapting to climate change.

\subsubsection{Hypothesis and objectives of the study}

Our main hypothesis is that there exist a viable control law of the diversity level of species or of selective disturbance targets that can respect the constrains on biodiversity and a diversity of ecosystem services (MFI???). To test this, we developed a simple theoretical model of multi-species/structure forest ecosystem dynamics. The model was parameterized for three species with different vital rates (growth, survival, reproduction) and competing for light between three vertical storeys (upper, mid, under). The model was used to predict annual Shannon diversity index of species, vertical structure, above-ground biomass, and timber extraction, under all the possible combinations of logging strategies affecting the three species and storeys every five years for 100 years.  Given the combinations, the algorithm from viability theory were used to deduce the the set of controls (including index of diversity of selective logging targets) that respect . We then compared the results with existing typoloigy of forests and practices used in forestry practices, in order to provide more relevant guidelines, that we discussed based on existing literature. 

\printbibliography

\end{document}